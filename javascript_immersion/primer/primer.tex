\documentclass{article}

\usepackage[T1]{fontenc}  % \textquotedbl

\usepackage{textcomp}     % \textquotesingle

\usepackage{lmodern}

\usepackage{csquotes} % \blockquote 

\usepackage{color}
\definecolor{lightgray}{rgb}{.95,.95,.95}
\definecolor{darkgray}{rgb}{.4,.4,.4}
\definecolor{purple}{rgb}{0.65, 0.12, 0.82}

\usepackage{upquote} % fixes single quotes in listings

\usepackage{listings}

\lstdefinelanguage{JavaScript}{
  keywords={typeof, new, true, false, catch, function, return, null, catch, switch, var, if, in, while, do, else, case, break},
  keywordstyle=\color{blue}\bfseries,
  ndkeywords={class, export, boolean, throw, implements, import, this},
  ndkeywordstyle=\color{darkgray}\bfseries,
  identifierstyle=\color{black},
  sensitive=false,
  comment=[l]{//},
  morecomment=[s]{/*}{*/},
  commentstyle=\color{purple}\ttfamily,
  stringstyle=\color{red}\ttfamily,
  morestring=[b]",
  morestring=[b]'
}

\lstset{
   language=JavaScript,
   backgroundcolor=\color{lightgray},
   extendedchars=true,
   basicstyle=\footnotesize\ttfamily,
   showstringspaces=false,
   showspaces=false,
   % numbers=left,
   % numberstyle=\footnotesize,
   % numbersep=9pt,
   tabsize=2,
   breaklines=true,
   showtabs=false,
   captionpos=b
}

\usepackage{hyperref}
\hypersetup{colorlinks=true,
	linkcolor=red} % use red colored links instead of weird boxes

\begin{document}

\title{JavaScript by Immersion}
\author{Brazos Valley Makers}

\maketitle

\section{About JavaScript}
JavaScript is a prototypal, weakly typed, dynamically typed scripting language created for Netscape by Brendan Eich in 10 days in 1995\footnote{\url{ https://www.w3.org/community/webed/wiki/A_Short_History_of_JavaScript}}. All of these characteristics have specific implications for JavaScript, and we will explore each of them and others as we go along.

\section{When in Doubt, Type it Out}
This primer will take a kinesthetic, learning by doing, approach to JavaScript. As such, we will need some things, most notably a JavaScript REPL (Read Eval Print Loop—in our case Node) and a browser with decent console capabilities (Chrome or Firefox will suffice).

\subsection{I Seriously Hope You Have a Browser Installed Already}
If not, there is no hope for you, but I heard they are serving refreshments somewhere.

\subsection{Installing Node}

\subsubsection{On a Mac}

\paragraph{If You Don’t Have Hombrew Installed}

\begin{lstlisting}
$ ruby -e "$(curl -fsSL\
 https://raw.githubusercontent.com/Homebrew/install/master/install)"
\end{lstlisting}

\paragraph{Installing Node with Homebrew}

\begin{lstlisting}
$brew install node
\end{lstlisting}

When installing node with homebrew, you will see the following message:
\blockquote{If you update npm itself, do NOT use the npm update command.
The upstream-recommended way to update npm is:
  npm install -g npm@latest
}

This is nothing to worry about. It simply means, that when you want to update the node package manager, do it like so:

\begin{lstlisting}
$npm install -g npm@latest
\end{lstlisting}

\subsubsection{On Windows}
Go to \url{https://nodejs.org/download/} and download the installer. Run the installer.

\section{Functions}
Simply put, a function performs one or more operations. Functions can exist on their own. Functions can be use to define objects. Functions can be used to give objects behavior.

\subsection{Functions Can Produce a Value}
At the command prompt in a terminal window (Type ``Terminal'' in Spotlight in OS X or Press Windows + R to bring up the Run box and type cmd.exe to open a terminal), type in node and hit Enter. You should be greeted with the Node prompt (a ``>''). Type in the following:

\begin{lstlisting}
function addOne (n)  { return n+1; }
\end{lstlisting}

And press the \emph{Enter} key.

Then type the following pressing enter at the end of the line:

\begin{lstlisting}
addOne(1);
\end{lstlisting}

You should see:

\begin{lstlisting}
2
>
\end{lstlisting}

From here on out, it is implied that you need to press the \emph{Enter} key after typing in code to the Node prompt.

\subsection{Functions Don't Need a Name}
Type:

\begin{lstlisting}
(function () { return 'You never even call me by my name.'; })();
\end{lstlisting}

\subsection{Functions Can be Function Parameters}
Type:

\begin{lstlisting}
function after () {
    console.log('after');
}

function before (callback) {
    console.log('before');
    callback();
}

before(after);
\end{lstlisting}

This feature is taken advantage of in Continuation Passing Style.

\section{Numbers}
Type:

\begin{lstlisting}
typeof 1;
typeof 1.1;
\end{lstlisting}

JavaScript is very egalitarian when it comes to numbers. It treats them all as floats.\footnote{\url{http://speakingjs.com/es5/ch11.html}} It will print numbers with nothing after the decimal place as integer numbers though.

Side note: \emph{typeof} is a unary operator, not a function, so you don't need parentheses to use it.

\subsection{Arithmetic}
Type:

\begin{lstlisting}
1 + 2 * 6;
(1 + 2) * 6;
\end{lstlisting}

Arithmetic operations are performed from left to right in precedence order. Expressions in parentheses are evaluated first.\footnote{\url{https://developer.mozilla.org/en-US/docs/Web/JavaScript/Reference/Operators/Operator_Precedence}}

\section{Strings}
Type the following at the Node prompt:

% TODO: How do we do typewriter quotes in TeX?
\begin{lstlisting}
'Hola';
"Mundo";
\end{lstlisting}

Strings are basically just text enclosed in quotation marks.\footnote{\url{https://developer.mozilla.org/en-US/docs/Web/JavaScript/Reference/Global_Objects/String}}

Type:
\begin{lstlisting}
'one' + 'two' + 'three';
\end{lstlisting}

As you can see, strings can be composed from smaller strings. This is called string concatenation.

Try:
\begin{lstlisting}
'hello world'.replace('world', 'college station');
\end{lstlisting}

JavaScript gives us all kinds of neat things we can do to string primitives courtesy of type coercion and the String constructor.\footnote{\url{https://developer.mozilla.org/en-US/docs/Web/JavaScript/Reference/Statements/var}}

Try:
\begin{lstlisting}
"Hook 'Em".toLowerCase();
'whoop'.toUpperCase();
\end{lstlisting}

A \emph{string primitive} is a value that is a string. A \emph{string literal} is literally \textquotedbl a string\textquotedbl. 

\end{document}